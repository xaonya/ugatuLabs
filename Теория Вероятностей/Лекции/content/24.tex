\section{Условное математическое ожидание}
$$(\Omega, F, P), \xi \in L^1(\Omega)$$
$G \in F, G - \sigma$-алгебра.
\begin{definition}[Условное Математическое Ожидание -- УМО]
Случайная величина $\xi$ относительно $\sigma$-алгебры $G$ называется случайной величиной $E(\xi | G)$, такой что:
\begin{enumerate}
\item $E(\xi | G) - G$-измерима
\item $\forall A \in G \Rightarrow \int\limits_A \xi dP = \int\limits_A E(\xi | G) dP$
\end{enumerate}
\end{definition}
\begin{remark}
 Для $\xi \notin L^1(\Omega)$ УМО не существует
\end{remark}
\subsection{Существование УМО}
Очевидно, что $\xi \geqslant 0$. Построим меру $\lambda(A) = \int\limits_A \xi dP, A \in G$, где $\lambda(A)$ - некоторая конечная мара на $G$ (т.к. $\xi \in L^1(\Omega)$)

В силу Теоремы Радона-Никодима имеем: мера $\lambda(A)$ абсолютно неприрывна относительно $p(A) : \lambda\ll p$, а значит, существует $p(\omega)$, такая что:
\begin{enumerate}
\item $p(\omega) - G$-измерима;
\item $\int\limits_A \xi dP = \int\limits_A p(\omega)dP \Rightarrow E(\xi | G)  = p(\omega)$.
\end{enumerate}
\begin{remark}
Из способа построения УМО следует, что $E(\xi | G)$ определено с точностью до множеств нулевой вероятности (из-за точности $p(\omega)$)
\end{remark}
\subsection{Свойства УМО}
\begin{enumerate}
\item $\xi = c = const \Rightarrow E(\xi | G) = c = const$ почти наверное
\begin{proof}
 $\xi = c = const$, а значит, измерима относительно любой $\sigma$-алгебры. Следовательно,
 $E(\xi | G) = const - G$-измерима. Тогда, при $A \in G$ имеем:
 $$\int\limits_A c dP = \int\limits_A \underbrace{E(\xi | G)}{=c} dP$$
\end{proof}
\item $\xi \leqslant \eta \Rightarrow E(\xi | G) \leqslant E(\eta | G)$ почти наверное
\begin{proof}
$\int\limits_A \xi dP = \int\limits_A (\eta | G) dP$, $\int\limits_A \xi dP = \int\limits_A (\eta | G) dP$

Из $\xi \leqslant \eta \Rightarrow \int\limits_A (\xi | G) dP \leqslant \int\limits_A (\eta | G) dP$, при чём оба подинтегральных выражения $G$-измеримы. Тогда

$\forall A \in G \Rightarrow E(\xi | G) \leqslant E(\eta | G)$
\end{proof}
\item $|E(\xi | G)| \leqslant E(|\xi| | G)$ почти наверное
\begin{proof}
$-|\xi|\leqslant \xi\leqslant |\xi| \Rightarrow E(-|\xi| | G)\leqslant E(\xi | G)\leqslant E(|\xi| | G)\Rightarrow -E(|\xi| | G)\leqslant E(\xi | G)\leqslant E(|\xi| | G)$
\end{proof}
\item $E(A \xi + B \eta | G) = A E(\xi | G) + B E(\eta | G)$
\begin{proof}
Обе стороны $G$-измеримы, проверим определение
$\int\limits_c (A\xi + B\eta) dP = \int\limits_c E(A\xi+B\eta | G) dP - G$-измерима

$\int\limits_c (A\xi + B\eta) dP = A\int\limits_c \xi dP + B\int\limits_c \eta dP = A\int\limits_c E(\xi | G) dP + B\int\limits_c E(\eta | G) dP = \int\limits_c (A E(\xi | G) + B E(\eta |G)) dP - G$-измерима.
\end{proof}
\item $G = \{\emptyset, \Omega\} \Rightarrow E(\xi | G) = E\xi$
\begin{proposition}
$$E\xi = \int\limits_\Omega \xi dP$$
$\forall A \subset G \Rightarrow \int\limits_A E(\xi | G) dP = \int\limits_A\xi dP$

В частности, $\int\limits_G E(\xi | G) dP = \int\limits_G\xi dP$

Если $G = \emptyset \Rightarrow 0 = 0$

Если $G = \Omega \Rightarrow \int\limits_\Omega E(\xi | G)dP = E\xi \Rightarrow E\xi = E\xi$
\end{proposition}
\item $\xi - G$-измерима, и $E(\xi | G) = \xi$ почти наверное.
\item $E(E(\xi | G) = E\xi$ почти наверное.
\item $E$
\end{enumerate}
Относительно приведённой $\sigma$-алгебры измеримы только постоянные случайные величины, следовательно, $E(\xi | G) = c$.

Возьмем $A\in\Omega \Rightarrow \int\limits_\Omega \xi dP = \int\limits_\Omega c dP = c \cdot p(\omega) = c \Rightarrow E\xi = c$
