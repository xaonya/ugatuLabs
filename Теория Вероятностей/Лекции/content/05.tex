\section{Классификация вероятностных мер}

\begin{definition}
  $F(x)$ -- абсолютно непрерывна, если её можно восстановить по производной $P(x)$
\end{definition}

\begin{example}
  $F(x) - F(a) = \int\limits_a^x P(y) dy$\\
  $F(x) - \text{функция распределения}, P(x) = \dfrac{d}{dx}F(x)$
\end{example}

\begin{definition}[Производная Радона-Никодима (Плотность)]
  Функция распределения $F(x)$ и соответствующая ей вероятностная мера $P(x)$ называются абсолютно-непрерывными, если $$P(B)=\int\limits_B P(y)dy$$
\end{definition}

В абсолютно непрерывном случае для задания вероятностной меры и соответсвующей ей функции распределения, достаточно задать плотность $P(y)$:
\begin{enumerate}
  \item $P(Y) \leq 0$
  \item $\int\limits_{-\infty}^{+\infty}P(y) dy = F(+\infty) = 1$
\end{enumerate}

\begin{definition}
  Непрерывная неубывающая функция распределенния $F(x)$ называется сингулярной, если $F'(x)=0$, при почти всех $x$.
\end{definition}

\begin{theorem}[Лебега о разложении произвольной функции распределенния]
  $$F(x)=p_1 F_d (x)+p_2 F_a (x)+p_3 F_s (x); p_1, p_2, p_3 \geq 0; p_1+p_2+p_3 = 1$$
  $F_d (x)$ -- дискретная функция распределения\\
  $F_a (x)$ -- абсолютно-непрерывная функция распределения\\
  $F_s (x)$ -- сингулярная функция распределения
\end{theorem}
