\section{Неравенство Чебышева, закон больших чисел}
\begin{definition}[Одинаково распределённые случайные величины]
$\xi_1$ и $\xi_2$ -- одинаково распределены, если $F_{\xi_1}(x)=F_{\xi_2}(x)$
\end{definition}
У одинаково распределённых случайных величин любые числовые характеристики одни и те же, так что во многих случаях можно считать, что одинаково распределённые случайные величины это есть копии одного и того же случайного эксперимента.
$$\bar{x}_n = \frac{\xi_1 + ... + \xi_n}{n}$$
Пусть $a = E\xi, \sigma^2 = D(\xi) < \infty$ -- конечные моменты I и II порядка.
$$\bar{\xi} = \frac{1}{\pi}\cdot\frac{1}{1+x^2}=P_\xi(x); E\xi = \frac{1}{\pi}\int\limits_R \frac{x}{1+x^2}dx$$
\begin{theorem}[Закон больших чисел для независимых одинаково распределённых случайных величин с конечной дисперсией]
Пусть $\xi_{1},..., \xi_{n}...$ -- последовательность независимых одинаково распределённых случайных величин.
$$\left.\begin{array}{r}
a = E\xi_{1}\\
\sigma^{2} = D(\xi_{1}) < \infty\\
\bar{x}_n = \frac{\xi_{1}+...+\xi_{n}}{n}\\
\end{array}\right\}\Rightarrow\lim\limits_{n\rightarrow\infty}\bar{x}_n=E\xi_1=a$$
\end{theorem}
\begin{proof}
Пусть $\sigma > 0$. Достаточно проверить что $P(|\bar{x}_n - a|>\sigma)\xrightarrow[n\rightarrow\infty]{}0$.\\
Вместо этого можно проверить сходимость $E(\bar{x}_n-a))\xrightarrow[n\rightarrow\infty]{}0$ (в силу неравенства Чебышева).\\
$E(\bar{x}_n\rightarrow a)^2)\xrightarrow[n\rightarrow\infty]{}0$\\
Найдём $E\bar{x}_n = \frac{1}{n}\sum\limits_{k=1}^n \xi_n=\frac{1}{n} a n = a$\\
$E(\bar{x}_n-a)^2=E(\bar{x}_n-E\bar{x}_n)^2=D(\bar{x}_n)$\\
$D(\bar{x}_n) = D\left(\frac{1}{n}\sum\limits_{k=1}^n \xi_k\right)=\frac{1}{n^2}D\left(\sum\limits_{k=1}^n \xi_k\right)=\frac{1}{n^2}\sum\limits_{k=1}^n \underbrace{D(\xi_k)}_{=\sigma^2}=\frac{1}{n^2}\sigma^2 n=\frac{\sigma^2}{n}\xrightarrow[n\rightarrow\infty]{}0$

Итого, для $L^2(\Omega)$ имеем: $\bar{x}_n\xrightarrow[n\rightarrow\infty]{}a$ и $D(\bar{x}_n)=\frac{\sigma^2}{n}$
\end{proof}
\begin{corollary}
\begin{enumerate}
\item Поскольку существует сходимость в $L^p(\Omega)\Rightarrow$ существует сходимость по вероятности, отсюда $P=\lim\limits_{n\rightarrow\infty}\bar{x}_n=a=E\xi_1$
\item Позже будет показано и доказано (при более слабых условиях) а точнее:\\
Если $E|\xi_1|<\infty, \xi_1,...,\xi_n$ -- независимо одинаково распределённые случайные величины, то $x_n\rightarrow a=E\xi_1$ почти наверно.
\item В процессе доказательства получили: $E\bar{x}_n = a, D(\bar{x}_n) = \frac{\sigma^2}{n}$
\item О законе статистической устойчивости случайных величин:\\
Если берём эксперимент $\xi$ и повторяем его многократно в одних и тех же условиях независимым образом, также возмём cобытие $A$, связанное с этим экспериментом. Тогде, если $n$ -- общее число повторений, то $\frac{n_A}{n}=p$ -- являющаяся постоянной относительная частота появления события $A$.
\item $P(A)=?$\\
$E\textbf{1}_A=P(A)$\\
Повторяем $A$ многократно в одних и тех же условиях.\\
$\textbf{1}_A^{(1)}, \textbf{1}_A^{(2)},...,\textbf{1}_A^{(n)}$ -- независимо одинаково распределённые случайные величины\\
$\frac{1}{n}\sum\limits_{k=1}^n \textbf{1}_A^{(k)}\rightarrow E\textbf{1}_A = P(A)$\\
$n_A=\sum\limits_{k=1}^n \textbf{1}_A^{(k)}$ -- число появления события $A$ в $n$ независимых испытаниях\\
$\frac{n_a}{n}=\frac{\textbf{1}_A^{(1)}+...+\textbf{1}_A^{(n)}}{n}\approx E\textbf{1}_A = P(A)$ согласно ЗБЧ.
\item $Ef(\xi)=?$\\
$\{\xi_{n}\} - n$ копий случайной величины $\xi$\\
$E\xi^{2} = \frac{\xi_{1}^{2}+...+\xi_{n}^{2}}{n}$\\
$Ef(\xi) = \frac{f(\xi_{1})+...+f(\xi_{n})}{n}$, где $f(*)$ -- детерминирована и $E|f(\xi)| < \infty$
\item $F_{\xi}(x) = p(\xi \leq x) = E\textbf{1}_{(\xi \leq x)} \approx \frac{\textbf{1}_{(\xi_{1} \leq x)} +...+\textbf{1}_{(\xi_{n} \leq x)}}{n} = \frac{1}{n}x_{n},$ где $x_{n}$ -- число появления $(\xi \leq x)$ в $n$ испытаниях.\\
 $F_{\xi}(x) = F_{n}^{*}(x) = \frac{1}{n} \sum\limits_{k=1}^{n} \textbf{1}(\xi_{k} \leq x)$ -- эмпирическая функция распределения.
\item $\int\limits_{a}^{b} f(x)dx = ?$ (Используем Метод Монте-Карло)\\
Пусть $\xi \sim [a, b]$ (равномерное распределение) $\Rightarrow P_\xi (x) = \left\{\begin{matrix} \frac{1}{b-a}, & x \in [a,b]\\0, & x \notin [a,b]\end{matrix}\right.$
Тогда, $Ef(\xi) = |$ замена переменных $| = \frac{1}{b-a} \int\limits_{a}^{b} f(x)dx$\\
$Ef(\xi) = \int\limits_\Omega f(\xi)dP = \int\limits_a^b \frac{1}{b-a} f(x)dx = \frac{1}{b-a}\int\limits_a^b f(x)dx =  \frac{f(\xi_{1})+...+f(\xi_{n})}{n} = \frac{1}{b-a} \int\limits_{a}^{b} f(x)dx$
\end{enumerate}
\end{corollary}
