\section{Дискретная вероятностная модель}
\begin{definition}[Случайная величина]
Cлучайный эксперимент, все возможные значения которого можно отождествить с $R$ называется \textbf{случайной величиной}.
\end{definition}
\begin{definition}[Случайный вектор]
Cлучайный эксперимент, все возможные значения которого можно отождествить с векторами называют \textbf{случайным вектором}.
\end{definition}
\begin{definition}[Случайный процесс]
Случайный эксперимент, все возможные значения которого можно считать функциями называется \textbf{случайным процессом}.
\end{definition}
\begin{definition}[Дискретная случайная величина]
Если всех возможных значений $(\xi, \eta, ...)$ случайной величины не более чем счётное множество, такая случайная величина называется \textbf{дискретной}.
\end{definition}
Пусть дана дискретная случайная величина $\xi$  и все её возможные значения $x_1,...x_n,...$.
\begin{definition}
Элементарные события $x_n$, происходящие в результате экспериментов $\xi$, называются исходами.
\end{definition}
$\Omega = \{\omega_1,...,\omega_n,...\}$ -- множество всех элементарных событий.\\
$A\leq\Omega$ -- случайное событие, $\emptyset$ -- невозможное событие.\\
Если происходит $A=\{\omega_1,...,\omega_k\}\Leftrightarrow$ происходит хотя бы одно из элементарных событий, из которых оно состоит.\\
\begin{definition}
Событие, которое происходит всегда, называется \textbf{достоверным} событием.
\end{definition}
Так как любые события являются множествами, над ними можно совершать те же действия, что и над множествами. \\
$A+B=A\cup B$ -- происходит или $A$ или $B$\\
$A\cdot B = AB = A\cap B$ -- происходят и $A$ и $B$\\
$\bar{A} = \Omega\backslash A$ -- противоположенное событие (происходит, когда не происходит $A$).\\
$F=\left\{A, \bigcup\limits_n A_n, \bigcap\limits_n A_n\right\}$ -- множество всех событий, где $F - \sigma$-алгебра, то есть, $\xi\rightarrow F$.
\paragraph{Вероятность события} Чтобы определить вероятность, достаточно знать вероятность элементарных событий $\omega_1,...,\omega_n,...$. Им можно сопоставить числа $p_1,...,p_n,...$, большие нуля, сумма которых равна единице.\\
Если $A\in F$ -- любое событие, то его вероятность можно посчитать следующим образом:
$$P(A)=\sum\limits_{k:\omega_k \in A} p_k, P(\emptyset)=0$$
Следовательно, $0\leqslant P(A)\leqslant 1, P(\Omega)=1$.
\begin{proposition}
$\forall A,B \Rightarrow P(A+B) = P(A)+P(B)-P(AB)$
\end{proposition}
\begin{proof}
\begin{multline*}
A+B=A\backslash B+B\backslash A+AB\Rightarrow P(A+B)=\\=\underbrace{P(A\backslash B)+P(AB)}_{=P(A)}+\underbrace{P(B\backslash A)+P(AB)}_{=P(B)}-P(AB)=\\=P(A)+P(B)-P(AB)
\end{multline*}
\end{proof}
$$P(A+B)\leqslant P(A)+P(B)\Rightarrow P\left(\bigcup\limits_n A_n\right)\leqslant\sum\limits_n P(A_n)$$
\begin{definition}[Несовместные события]
События, которые не могут произойти одновременно. ($A$ и $B$ несовместны $\Leftrightarrow AB = 0$)
\end{definition}
Для несовместных событий вероятность суммы равна сумме вероятностей
$$P(A+B)=\sum\limits_{k:\omega_k \in A+B} p_k = \sum\limits_{k:\omega_k \in A} p_k+\sum\limits_{k:\omega_k \in B} p_k = P(A)+P(B)$$
Если $A_1,...,A_n,...$ -- последовательность несовместных событий, то, в силу теоремы о свойствах положительных рядов, $P\left(\bigcup\limits_n A_n\right)=\sum\limits_n P(A_n)$. Также верно что $P(\bar{A})=1-P(A)$, так как  $P(A)+P(\bar{A})=P(A+\bar{A})=P(\Omega)=1$\\
Таким образом по случайному эксперименту $\xi$ строится \textbf{дискретное вероятностное пространство} $\xi\rightarrow(\Omega,F,P)$, где $P(A)$ -- \textbf{вероятностная мера} на $F, P(\Omega)=1$.
\subsection{Закон статистической устойчивости}
Если есть случайный эксперимент $\xi$ и с ним связано событие $A$, то можно многократно повторить $\xi$ в одних и тех же условиях и посчитать относительную частоту случайного события $\frac{n_a}{n}\rightarrow P(A)$, где $n_a$ -- число экспериментов, в которых произошло событие $A$, а $n$ -- общее число экспериментов.\\
Теория вероятности изучает только те случайные события, для которых справедлив принцип статистической устойчивости.
\subsection{Закон распределения дискретной случайной величины}
Построим таблицу зависимости $p$ от $\xi$ по правилу $p_n=P(\xi=x_n)=P(\omega_n)$
\begin{tabular}{c|ccccc}
$\xi$ & $x_1$ & $x_2$ & $x_3$ & $...$ & $x_n$\\
\hline
$p$ & $p_1$ & $p_2$ & $p_3$ & $...$ & $p_n$
\end{tabular}
Эта таблица -- закон распределения дискретной случайной величины -- содержит ту же информацию, что и описание дискретного вероятностного пространства на вероятностной мере.

Каждое дискретное пространство предоставляет всю информацию для дискретной случайной величины.
\subsection{Классическая модель. Классическое определение вероятности}
Под классической моделью понимается дискретное вероятностное пространство $(\Omega,F,P)$ и соответствующий циклический эксперимент,, удовлетворяющий следующим свойствам:
\begin{enumerate}
 \item Число элементарных событий конечно
 \item Все исходы $\{\omega_n\}$ равновероятны, то есть $p_1=p_2=...=p_n=\frac{1}{n}$
\end{enumerate}
$$P(A)=\sum\limits_{k:\omega_k\in A}p_k=\sum\limits_{k:\omega_k\in A}\frac{1}{n}=\frac{1}{n}\underbrace{\sum\limits_{k:\omega_k\in A}1}_{=n(A)}=\frac{n(A)}{n}$$
$n(A)$ -- число выпадений события $A$, тогда $P(A)=\frac{n(a)}{n}$, где $n$ -- общее число экспериментов.
\subsection{Элементы комбинаторики}
Если сложное действие состоит из  $k$ более простых, каждое из которых можно выполнить $n_1,...,n_k$ способами, то общее число возможных способов выполнить сложное действие равно $N=n_1\cdot...\cdot n_k$.
\begin{definition}
\textbf{Перестановкой} из $n$ элементов (например, чисел от $1$ до $n$) называется всякий упорядоченный набор из этих элементов. Обозначается факториалом. Перестановка также является размещением из $n$ элементов по $n$.
\end{definition}
\begin{definition}
\textbf{Сочетанием} из $n$ по $k$ элементов называется набор $k$ элементов, выбранных из данных $n$ элементов. Обозначение $C_n^k$. Наборы, не отличающиеся составом элементов, считаются одинаковыми. Этим сочетания отличаются от размещений.
\end{definition}
\begin{definition}
\textbf{Размещением} из $n$ по $k$ элементов называется упорядоченный набор из $k$ различных элементов некоторого $n$-элементного множества. Обозначается $A_n^k = C_n^k k!$.
\end{definition}
