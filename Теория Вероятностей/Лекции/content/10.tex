\section{Случайные величины, простые случайные величины, $\sigma$-алгебры, порождённые случайными величинами. Независимость событий. Формулы полной вероятности и Байеса}

% TODO:

%\section{Условные вероятности. Независимость событий. Формулы полной вероятности и Байеса}
%\begin{definition}[Условная вероятность]
%Пусть есть два события -- $A$ и $B$.\\
%Тогда \textbf{условная вероятность} наступления события $A$ при %условии наступления события $B$ расчитывается по следующей формуле:
%$$P(A|B)=\frac{P(A\cdot B)}{P(B)}, P(B) \ne 0$$
%\end{definition}
%\begin{corollary}
%$P(A\cdot B) = P(A|B)\cdot P(B)$
%\end{corollary}
%Продолжение следует ...

