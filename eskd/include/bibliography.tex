\usepackage[
  style=numeric-comp,
%   natbib=true,
%   backend=biber
% ]{biblatex}
  % backend=biber,% движок
  bibencoding=utf8,% кодировка bib файла
  sorting=none,% настройка сортировки списка литературы
  % style=gost-numeric,% стиль цитирования и библиографии (по ГОСТ)
  % language=autobib,% получение языка из babel/polyglossia, default: autobib % если ставить autocite или auto, то цитаты в тексте с указанием страницы, получат указание страницы на языке оригинала
  % autolang=other,% многоязычная библиография
  % clearlang=true,% внутренний сброс поля language, если он совпадает с языком из babel/polyglossia
  defernumbers=true,% нумерация проставляется после двух компиляций, зато позволяет выцеплять библиографию по ключевым словам и нумеровать не из большего списка
  sortcites=true,% сортировать номера затекстовых ссылок при цитировании (если в квадратных скобках несколько ссылок, то отображаться будут отсортированно, а не абы как)
  doi=false,% Показывать или нет ссылки на DOI
  isbn=false,% Показывать или нет ISBN, ISSN, ISRN
]{biblatex}

\newcounter{refs}
\AtEveryBibitem{\stepcounter{refs}}

\makeatletter
  \AtEndDocument{%
    \immediate\write\@mainaux{%
      \string\gdef\string\totrefs{\number\value{refs}}%
    }%
  }
\makeatother
