\section{Матрицы}
\subsection{Основные определения}

\begin{definition}
  Матрицей размеров $m \times n$ над множеством действительных чисел $\mathds{R}$
  называется прямоугольная таблица из $m \cdot n$ вещественных чисел, имеющая $m$ строк и $n$ столбцов:

  \[ \begin{array}{cccc}
    a_{11}& a_{12} &\ldots & a_{1n}\\
    a_{21}& a_{22} &\ldots & a_{2n}\\
    \vdots& \vdots &\ddots & \vdots\\
    a_{m1}& a_{m2} &\ldots & a_{mn}
  \end{array} \]

  где $a_{ij} \in \mathds{R}, i = \overline{1, m}$ - номер строки,
    $j = \overline{1, n}$ - номер столбца, $a_{i,j}$ - элементы матрицы, $m$ и $n$ - порядки матрицы.
  В этом случае говорят, что рассматриваемая матрица размера $m \times n$.
  Если $m = n$, то матрица называется квадратной, а число $m = n$ - её порядком.
\end{definition}

Для изображения матрицы применяются либо круглые скобки, либо сдвоенные прямые:
  \[ \left( \begin{array}{cccc}
    a_{11}& a_{12} &\ldots & a_{1n}\\
    a_{21}& a_{22} &\ldots & a_{2n}\\
    \vdots& \vdots &\ddots & \vdots\\
    a_{m1}& a_{m2} &\ldots & a_{mn}
  \end{array} \right)
  \text{ или }
  \begin{Vmatrix}
    a_{11}& a_{12} &\ldots & a_{1n}\\
    a_{21}& a_{22} &\ldots & a_{2n}\\
    \vdots& \vdots &\ddots & \vdots\\
    a_{m1}& a_{m2} &\ldots & a_{mn}
    \end{Vmatrix}
  \]

Для краткого обозначения матрицы используются либо заглавные латинские буквы ($A$, $B$, $C$, \dots)
либо символы ($a_{ij}$), $||a_{ij}||$, указывающие обозначение элементов матрицы;
либо используется запись
$A = (a_{i,j})(i = \overline{1,m}$, $j = \overline{1,n})$.

Множество всех матриц размера $m \times n$ бозначается $\mathds{R}_{m \times n} \equiv \mathds{R}_{m,n}$.

\subsubsection{Частные случаи матриц}
\begin{enumerate}
  \item Если $m = n$, то матрица называется квадратной.
    Её диагональ $a_{1,1}, a_{2,2}, \dots, a_{n,n}$ называется главной диагональю, а $a_{n1}, a_{n-1,2}, \dots, a_{1n}$ – побочной диагональю.

  \item Диагональная матрица – это матрица, у которой все ненулевые элементы находятся на главной диагонали, т.е.
    $A = \left( \begin{array}{cccc}
      a_{11}& 0 &\ldots & 0\\
      0& a_{22} &\ldots & 0\\
      \vdots& \vdots &\ddots & \vdots\\
      0& 0 &\ldots & a_{mn}
    \end{array} \right)$.

  \item Диагональная матрица вида
    $\left( \begin{array}{cccc}
      a& 0 &\ldots & 0\\
      0& a &\ldots & 0\\
      \vdots& \vdots &\ddots & \vdots\\
      0& 0 &\ldots & a
    \end{array} \right)$ называется скалярной.

  \item Скалярная матрица с единичными элементами на главной диагонали называется единичной.
    Обозначается $E$ или $E_{n}$, где n - ee порядок.

  \item Матрица размера $m \times n$, у которой все элементы равны нулю, называется нулевой и обозначается $O_{m,n}$.

  \item Если $m = 1$, то матрица называется строчной, или матрица-строка, или строка.
    Если $n = 1 \rightarrow$ столбцовая, или матрица-столбец, или просто столбец.
\end{enumerate}

\begin{definition}
  Две матрицы называются равными, если эти матрицы имеют одинаковые порядки и их соответствующие элементы совпадают.
\end{definition}

\subsection{Операции над матрицами и их свойства}
\begin{definition}
  Суммой матриц $A$ и $B \in \mathds{R}_{m,n}$ (т.е. имеющих одинаковые порядки) называется матрица $C \in \mathds{R}_{m,n}:c_{ij}=a_{ij}+b_{ij}, i = \overline{1,m}, j = \overline{1,n}$.
\end{definition}
\noindent \textbf{Обозначение:} $C = A + B$.

\begin{example}
  \[ \begin{pmatrix}
    1& 2& 3\\
    -2& -3& 1
  \end{pmatrix} + \begin{pmatrix}
    2& 0& -1\\
    3& 2& -1
  \end{pmatrix} = \begin{pmatrix}
    3& 2& 2\\
    1& -1& 0
  \end{pmatrix} \]
\end{example}

\begin{properties}[сложения матицы]
  $\\$
  \begin{enumerate}
    \item Коммутативность сложения, т.е. $\forall A,B \in \mathds{R}_{m,n}$ справедливо $A+B=B+A $.
    \item Ассоциативность сложения, т.е. $\forall A,B,C \in \mathds{R}_{m,n}$ справедливо $(A+B)+C=A+(B+C)$.
    \item $\forall A \in \mathds{R}_{m,n}, A+O_{m,n}=O_{m,n}+A=A$.
    \item $\forall A \in \mathds{R}_{m,n} \exists! B \in \mathds{R}_{m,n} : A+B=B+A=O_{m,n}$. При этом, если $A=(a_{ij})$, то $b_{i,j}=-a_{ij}$. Матрица $B$ называется противоположной к $A$ и обозначается $-A$.

    Доказательство свойств провести самостоятельно прямыми вычисления-ми.
  \end{enumerate}
\end{properties}

\begin{definition}
  Произведением элемента $\alpha \in \mathds{R}$ на матрицу $A \in \mathds{R}_{m,n}$ называется матрица $C \in \mathds{R}_{m,n} : c_{ij}=\alpha a_{ij}, i = \overline{1, m}, j = \overline{1, n}$.
\end{definition}
\noindent \textbf{Обозначение:} $C=\alpha A$.

\noindent Операция, сопоставляющая $\alpha$ и $A$ их произведению $C=\alpha A$ называется умножением числа на матрицу.

\begin{properties}[умножения матрицы на число]
$\\ \forall \alpha, \beta \in \mathds{R}_{m,n}$ выполняется:
\begin{enumerate}
  \item $\alpha(\beta A) = (\alpha \beta)A$,
  \item $(\alpha + \beta)A = \alpha A + \beta A$,
  \item $\alpha(A+B)=\alpha A + \alpha B$,
  \item $1 \cdot A = A $.

  Доказательство свойств - самостоятельно прямыми вычисления-ми.
\end{enumerate}
\end{properties}

\textbf{Замечание.} Разность $A-B$ двух прямоугольных матриц $A$ и $B \in \mathds{R}_{m,n}$ определяется равенством $A-B=A+(-1)B=A+(-B)$.
\begin{definition}
  Произведением матриц $A=(a_{ij})$ размера $m \times n$ и $B=(b_{ij})$ размера $n \times p$ называется матрица $C=(c_{ij})$ размеров $m \times p$ такая, что каждый элемент $c_{ij}= \sum \limits_{k=1}^n a_{ik}b_{kj}$.
\end{definition}
\noindent \textbf{Обозначение:} $C=A \cdot B \equiv AB$.
Операция произведения $A$ и $B$ называется умножением этих матриц.

Из определения следует, что элемент матрицы $A$ и $B$ , стоящий в $i$–ой строке и $j$–ом столбце, равен сумме произведений элементов $i$–ом столбце, равен сумме произведений элементов $A$ на $j$–ый столбец матрицы $B$.
\begin{example}
  \begin{enumerate} $\\$
    \item $A= \begin{pmatrix}
      1& 3\\
      -2& 0
    \end{pmatrix}$, $\begin{pmatrix}
      -1& 3& 6\\
      2& -8& 7
    \end{pmatrix} \rightarrow AB=\begin{pmatrix}
      3& -13& 20\\
      3& -9& -18
    \end{pmatrix}$,
    \item $A= \begin{pmatrix}
      1& 2
    \end{pmatrix}$, $\begin{pmatrix}
      -1\\
      0
    \end{pmatrix} \rightarrow AB=\begin{pmatrix}
      -1
    \end{pmatrix} , BA=\begin{pmatrix}
      -1& -2\\
      0& 0
    \end{pmatrix}$.
  \end{enumerate}
\end{example}

Таким образом, две матрицы можно перемножать, если число столбцов матрицы $A$ равно числу строк матрицы $B$. Тогда матрица $A$ называется согласованной с $B$. Из согласованности $A$ c $B$ не следует согласованность $B$ с $A$. Если даже условие согласования выполняется, то в общем случае $AB \neq BA$.
\begin{properties}[умножения матриц]
  \begin{enumerate} $\\$
    \item Ассоциативность умножения матриц, т.е. $\forall A \in \mathds{R}_{m,n}, B \in \mathds{R}_{n,p}, C \in \mathds{R}_{p,l}$ справедливо $(AB)C=A(BC)$.


    \begin{proof}Из определения 5 следует, что элемент $d_{it}$ матрицы $(AB)C$ равен $d_{it} = \sum \limits_{k=1}^p(\sum \limits_{j=1}^n a_{ij}b_{jk})c_{kt}$, а элемент $\overline{d_{it}}$ матрицы $A(BC)$ равен $\overline{d_{it}} = \sum \limits_{j=1}^n a_{ij}(\sum \limits_{k=1}^p b_{jk}c_{kt})$. Равенство $d_{it}=\overline{d_{it}}$ следует из возможности изменения порядка суммирования.\end{proof}
    \item Дистрибутивность сложения относительно умножения, т.е. $\\ A(B+C) = AB +AC, \forall A \in \mathds{R}_{m,n}, B,C \in \mathds{R}_{n,p}. \\ (A + B)C = AC + BC, \forall A,B \in \mathds{R}_{m,n}, C \in \mathds{R}_{n,p}$.

    \begin{proof}
      Доказательство следует из определения суммы и произведения матриц.
    \end{proof}
    \item $E_{m}A = AE_{n} = A, \forall A \in \mathds{R}_{m,n }.$

    \begin{proof}
      Пусть $B=(b_{ij}):B=AE_{n}$, и $C=E_{m}A$. Тогда $b_{ij} = \sum \limits_{k=1}^n a_{ik} \delta_{kj} = a_{ij} \delta_{jj} = a_{ij}$. Здесь $\delta_{ij}= \begin{cases} 1,i=j \\ 0,i \neq j \end{cases}$ - символ Кронекера. $\\ c_{ij} = \sum \limits_{k=1}^m \delta_{ik}a_{kj}=a_{ij}$.
    \end{proof}
    \item $(\alpha E_{m})A= \alpha A = A(\alpha E_{n}), \forall A \in \mathds{R}_{m,n}, \forall \alpha \in \mathds{R}$.
    \item $\forall A \in \mathds{R}_{m,n}, O_{p,m} \cdot A = O_{p,n}, A \cdot O_{n,p} = O_{m,p}.$
    \begin{proof}
    Доказательство свойств 4 и 5 проводится аналогично свойству 3.
  \end{proof}

    \item $\forall A \in \mathds{R}_{m,n}, \forall B \in \mathds{R}_{n,p}, \forall \alpha \in \mathds{R} \rightarrow \alpha(AB) = (\alpha A)B = A(\alpha B)$.
  \end{enumerate}
\end{properties}
\textbf{Замечание.} В общем случае произведение матриц не коммутативно. Например, \[ \begin{pmatrix} 0 & 1\\ 0 & 0 \end{pmatrix} \begin{pmatrix} 0 & 0\\ 1 & 0 \end{pmatrix} \neq \begin{pmatrix} 0 & 0\\ 1  & 0 \end{pmatrix} \begin{pmatrix} 0 & 1\\ 0 & 0 \end{pmatrix} \]
  Но из свойств 4 и 5 $\rightarrow$ умножение квадратной матрицы на $E$ и $O$ коммутирует. Также коммутирует умножение квадратной матрицы на ска-лярную.

\subsection{Блочные матрицы.}
Пусть матрица $A$ при помощи горизонтальных и вертикальных прямых разбита на отдельные прямоугольные клетки, каждая из которых является матрицей меньших раз-меров и называется блоком исходной матрицы. В этом случае $A$ рассматривается как некоторая новая, блочная матрица $A = ||A_{\alpha \beta}||$, элементами которой являются блоки $||A_{\alpha \beta}||$ указанной матрицы ($A_{\alpha \beta}$– элементы матрицы, поэтому $A$ заглавное). Здесь $\alpha$ – номер блочной строки, $\beta$ – столбца.

Например, если \[A = \begin{pmatrix}
  \begin{tabular}{l | l}
    $a_{11}$ $a_{12}$ $a_{13}$&$a_{14}$ $a_{15}$\\ \hline
    $a_{21}$ $a_{22}$ $a_{23}$ & $a_{24}$ $a_{25}$ \\
    $a_{31}$ $a_{32}$ $a_{33}$ & $a_{34}$ $a_{35}$
    \end{tabular}
\end{pmatrix} \text{, то } A_{11} = \begin{pmatrix}
  a_{11} & a_{12} & a_{13}
\end{pmatrix}, A_{12} = \begin{pmatrix}
  a_{14} & a_{15}
\end{pmatrix},\]
\[ A_{21} = \begin{pmatrix}
  a_{21} & a_{22} & a_{23} \\
  a_{31} & a_{32} & a_{33}
\end{pmatrix}, A_{22} = \begin{pmatrix}
  a_{24} & a_{25} \\
  a_{34} & a_{35}
\end{pmatrix}
  \].

  Замечательным является факт, что операции с блочными матрицами совер-шаются по тем же правилам, что и с обычными, только в роли элементов вы-ступают блоки. Действительно, если
  $A=\begin{Vmatrix}
    A_{\alpha \beta}
  \end{Vmatrix}$, то $\lambda A = \begin{Vmatrix}
\lambda a_{ij}
  \end{Vmatrix} = \begin{Vmatrix}
    \lambda A_{\alpha \beta}
  \end{Vmatrix}$, где $\begin{Vmatrix}
    \lambda A_{\alpha \beta}
  \end{Vmatrix}$ вычисляется по обычному правилу умножения матрицы на число. Анало-гично, если $A$ и $B$ имеют одинаковые порядки и одинаковым образом разбиты на блоки, то сумме $A +B$ отвечает блочная матрица $C=\begin{Vmatrix}
  C_{\alpha \beta}
  \end{Vmatrix} : C_{\alpha \beta} = A_{\alpha \beta} + B_{\alpha \beta}$.

  Для умножения $A \in \mathds{R}_{m,n}$ на $B \in \mathds{R}_{n,p}$ необходимо согласовать их разбиение на блоки, т.е. число столбцов каждого блока $A_{\alpha \beta}$ должно быть равно числу строк блока $B_{\beta \gamma}$. Тогда $C_{\alpha \beta} = \sum \limits_{\gamma} A_{\alpha \gamma} B_{\gamma \beta}$.
  \begin{proof}
Для доказательства необходимо расписать правую и левую части в терминах обычных элементов матриц $C,A$ и $B$.Пусть разбиение матриц проведено следующим образом:
  $1 \leq m_{1} < m_{2} < \dotso < m_{k} = m, 1 \leq p_{1} < p_{2} < \dotso < p{q} = p.\\$
  Если $i \in m_{\alpha}, j \in p_{\beta}$, то $c_{ij}=C_{\alpha \beta}$ и $c_{ij}=\sum \limits_{t=1}^n a_{it}b_{tj} = \sum \limits_{t=1}^{n_{1}} + \sum \limits_{t=n_{1}+1}^{n_{2}} + \dotso + \sum \limits_{t=n_{l-1}+1}^{n_{l}}$, откуда следует, что $ C_{\alpha \beta} = A_{\alpha 1}B_{1 \beta} + \dotso + A_{\alpha l}B_{l \beta}$, что и требовалось доказать.
\end{proof}

\begin{example}
  Пусть $A=\begin{pmatrix}
    \begin{tabular}{l | l}
      3 & 0 -1\\
      2 & 1 0\\ \hline
      -6 & 0 3
    \end{tabular}
  \end{pmatrix}, B=\begin{pmatrix}
    \begin{tabular}{l | l}
      1 2 & 0 1 \\ \hline
      0-3 & 3 0 \\
      1 0 & 0 0
    \end{tabular}
  \end{pmatrix}$, т.е. $\\A=\begin{pmatrix}
    A_{11} & A_{12} \\
    A_{21} & A_{22}
  \end{pmatrix}, B=\begin{pmatrix}
    B_{11} & B_{12} \\
    B_{21} & B_{22}
  \end{pmatrix},$ где $A_{11}=\begin{pmatrix}
    3 \\
    2
  \end{pmatrix}, A_{12}=\begin{pmatrix}
    0 & -1 \\
    1 & 0
  \end{pmatrix}, A_{21}=\begin{pmatrix}
    -6
  \end{pmatrix}, A_{11}=\begin{pmatrix}
    0 & 3
  \end{pmatrix}, B_{11}=\begin{pmatrix}
  1 & 2
  \end{pmatrix}, B_{12}=\begin{pmatrix}
    0 & 1
  \end{pmatrix}, B_{11}=\begin{pmatrix}
    0 & -3 \\
    1 & 0
  \end{pmatrix}, B_{11}=\begin{pmatrix}
    3 & 0\\
    0 & 0
  \end{pmatrix}.$
  Тогда $C_{11} = A_{11}B_{11} + A_{12}B_{21} = \begin{pmatrix}
    3 & 6 \\
    2 & 4
  \end{pmatrix} + \begin{pmatrix}
    -1 & 0 \\
    0 & -3
  \end{pmatrix} = \begin{pmatrix}
    2 & 6 \\
    2 & 1
  \end{pmatrix}.$ Аналогично находятся остальные $C_{\alpha \beta}.$ В результате получаем $AB=\begin{pmatrix}
    \begin{tabular}{l | l}
      2 6 & 0 3 \\
      2 1 & 3 2 \\ \hline
      -3 -12 & 0 -6
    \end{tabular}
  \end{pmatrix}$.
\end{example}
В качестве применения блочных матриц рассмотрим
\begin{definition}
  Прямой суммой квадратных матриц $A$ и $B$ порядков $m$ и $n$ соответственно называется квадратная матрица $C$ порядка $\\m+n: C = \begin{pmatrix}
    A & 0 \\
    0 & B
  \end{pmatrix}.$
\end{definition}
\noindent \textbf{Обозначение:} $C=A \bigoplus B$.
\begin{properties}[прямой суммы]
\begin{enumerate} $ $
  \item $(A \bigoplus B) \bigoplus C = A \bigoplus (B \bigoplus C)$.
  \item $A \bigoplus B \neq B \bigoplus A$.
  \item $(A_{m} \bigoplus A_{n}) + (B_{m} \bigoplus B_{n}) = (A_{m} + A_{n}) \bigoplus (B_{m} + B_{n})$.
  \item $(A_{m} \bigoplus A_{n})(B_{m} \bigoplus B_{n}) = (A_{m}B_{m}) \bigoplus (A_{n}B_{n})$.
\end{enumerate}
Доказательство – самостоятельно.
\end{properties}