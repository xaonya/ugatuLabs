\section{Матрицы}
\subsection{Основные определения}

\begin{definition}
  Матрицей размеров $m \times n$ над множеством действительных чисел $\mathds{R}$
  называется прямоугольная таблица из $m \cdot n$ вещественных чисел, имеющая $m$ строк и $n$ столбцов:

  \[ \begin{array}{cccc}
    a_{11}& a_{12} &\ldots & a_{1n}\\
    a_{21}& a_{22} &\ldots & a_{2n}\\
    \vdots& \vdots &\ddots & \vdots\\
    a_{m1}& a_{m2} &\ldots & a_{mn}
  \end{array} \]

  где $a_{ij} \in \mathds{R}, i = \overline{1, m}$ - номер строки,
    $j = \overline{1, n}$ - номер столбца, $a_{i,j}$ - элементы матрицы, $m$ и $n$ - порядки матрицы.
  В этом случае говорят, что рассматриваемая матрица размера $m \times n$.
  Если $m = n$, то матрица называется квадратной, а число $m = n$ - её порядком.
\end{definition}

Для изображения матрицы применяются либо круглые скобки, либо сдвоенные прямые:
  \[ \left( \begin{array}{cccc}
    a_{11}& a_{12} &\ldots & a_{1n}\\
    a_{21}& a_{22} &\ldots & a_{2n}\\
    \vdots& \vdots &\ddots & \vdots\\
    a_{m1}& a_{m2} &\ldots & a_{mn}
  \end{array} \right)
  \text{ или }
  \begin{Vmatrix}
    a_{11}& a_{12} &\ldots & a_{1n}\\
    a_{21}& a_{22} &\ldots & a_{2n}\\
    \vdots& \vdots &\ddots & \vdots\\
    a_{m1}& a_{m2} &\ldots & a_{mn}
    \end{Vmatrix}
  \]

Для краткого обозначения матрицы используются либо заглавные латинские буквы ($A$, $B$, $C$, \dots)
либо символы ($a_{ij}$), $||a_{ij}||$, указывающие обозначение элементов матрицы;
либо используется запись
$A = (a_{i,j})(i = \overline{1,m}$, $j = \overline{1,n})$.

Множество всех матриц размера $m \times n$ бозначается $\mathds{R}_{m \times n} \equiv \mathds{R}_{m,n}$.

\subsubsection{Частные случаи матриц}
\begin{enumerate}
  \item Если $m = n$, то матрица называется квадратной.
    Её диагональ $a_{1,1}, a_{2,2}, \dots, a_{n,n}$ называется главной диагональю, а $a_{n1}, a_{n-1,2}, \dots, a_{1n}$ – побочной диагональю.

  \item Диагональная матрица – это матрица, у которой все ненулевые элементы находятся на главной диагонали, т.е.
    $A = \left( \begin{array}{cccc}
      a_{11}& 0 &\ldots & 0\\
      0& a_{22} &\ldots & 0\\
      \vdots& \vdots &\ddots & \vdots\\
      0& 0 &\ldots & a_{mn}
    \end{array} \right)$.

  \item Диагональная матрица вида
    $\left( \begin{array}{cccc}
      a& 0 &\ldots & 0\\
      0& a &\ldots & 0\\
      \vdots& \vdots &\ddots & \vdots\\
      0& 0 &\ldots & a
    \end{array} \right)$ называется скалярной.

  \item Скалярная матрица с единичными элементами на главной диагонали называется единичной.
    Обозначается $E$ или $E_{n}$, где n - ee порядок.

  \item Матрица размера $m \times n$, у которой все элементы равны нулю, называется нулевой и обозначается $O_{m,n}$.

  \item Если $m = 1$, то матрица называется строчной, или матрица-строка, или строка.
    Если $n = 1 \rightarrow$ столбцовая, или матрица-столбец, или просто столбец.
\end{enumerate}

\begin{definition}
  Две матрицы называются равными, если эти матрицы имеют одинаковые по-рядки и их соответствующие элементы совпадают.
\end{definition}

\subsection{Операции над матрицами и их свойства.}
\begin{definition}
  Суммой матриц $A$ и $B \in \mathds{R}_{m,n}$ (т.е. имеющих одинаковые порядки) называется матрица $C \in \mathds{R}_{m,n}:c_{ij}=a_{ij}+b_{ij}, i = \overline{1,m}, j = \overline{1,n}$.
\end{definition}
\textbf{Обозначение:} $C = A + B$.

\begin{example}
  \[ \begin{pmatrix}
    1& 2& 3\\
    -2& -3& 1
  \end{pmatrix} + \begin{pmatrix}
    2& 0& -1\\
    3& 2& -1
  \end{pmatrix} = \begin{pmatrix}
    3& 2& 2\\
    1& -1& 0
  \end{pmatrix} \]
\end{example}

\begin{properties}[сложения матриц]
  \begin{enumerate}

  \end{enumerate}
\end{properties}
